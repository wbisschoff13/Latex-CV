%!TEX TS-program = xelatex
%!TEX encoding = UTF-8 Unicode
% Awesome CV LaTeX Template for Cover Letter
%
% This template has been downloaded from:
% https://github.com/posquit0/Awesome-CV
%
% Authors:
% Claud D. Park <posquit0.bj@gmail.com>
% Lars Richter <mail@ayeks.de>
%
% Template license:
% CC BY-SA 4.0 (https://creativecommons.org/licenses/by-sa/4.0/)
%


%-------------------------------------------------------------------------------
% CONFIGURATIONS
%-------------------------------------------------------------------------------
% A4 paper size by default, use 'letterpaper' for US letter
\documentclass[12pt, a4paper]{awesome-cv}

\def\sidemargin{2.4cm}
\def\topbotmargin{2.4cm}
\geometry{top=\topbotmargin, bottom=\topbotmargin, left=\sidemargin, right=\sidemargin}
\fontdir[fonts/] % Specify the location of the included fonts

\usepackage{pagecolor}
\definecolor{page}{HTML}{fefdfa}
\pagecolor{page}


% Specify the location of the included fonts
\fontdir[fonts/]

% Color for highlights
% Awesome Colors: awesome-emerald, awesome-skyblue, awesome-red, awesome-pink, awesome-orange
%                 awesome-nephritis, awesome-concrete, awesome-darknight
% \colorlet{awesome}{awesome-red}
% Uncomment if you would like to specify your own color
\definecolor{awesome}{HTML}{559CC7}

% Colors for text
% Uncomment if you would like to specify your own color
\definecolor{darktext}{HTML}{333B42}
\definecolor{text}{HTML}{080808}
\definecolor{graytext}{HTML}{555E66}
\definecolor{lighttext}{HTML}{8B9297}


% Set false if you don't want to highlight section with awesome color
\setbool{acvSectionColorHighlight}{true}

% If you would like to change the social information separator from a pipe (|) to something else
\renewcommand{\acvHeaderSocialSep}{\quad\textbar\quad}

\renewcommand{\acvHeaderAfterNameSkip}{2.8mm}
\renewcommand{\acvHeaderAfterPositionSkip}{0.8mm}
\renewcommand{\acvHeaderAfterAddressSkip}{0.1mm}
\renewcommand{\acvSectionTopSkip}{2mm}

%-------------------------------------------------------------------------------
%	PERSONAL INFORMATION
%	Comment any of the lines below if they are not required
%-------------------------------------------------------------------------------
% Available options: circle|rectangle,edge/noedge,left/right
% \photo[circle,noedge,left]{./img/profile}
\name{Werner}{Bisschoff}
\position{Embedded Software Engineer}
\address{Cape Town, Western Cape, South Africa}

\email{werner@bisschoff.dev}
% \mobile{067 081 7719}
\whatsapp{067 081 7719}
% \homepage{bisschoff.dev} %TODO: add website
% \github{wbisschoff13}
\linkedin{wbisschoff13}
% \gitlab{gitlab-id}
% \stackoverflow{SO-id}{SO-name}
% \twitter{@twit}
% \skype{skype-id}
% \reddit{reddit-id}
% \medium{madium-id}
% \googlescholar{googlescholar-id}{name-to-display}
%% \firstname and \lastname will be used
% \googlescholar{googlescholar-id}{}
% \extrainfo{extra informations}

% \quote{``Be the change that you want to see in the world."}


%-------------------------------------------------------------------------------
%	LETTER INFORMATION
%	All of the below lines must be filled out
%-------------------------------------------------------------------------------
% The company being applied to
% \recipient
%   {Company Recruitment Team}
%   {Google Inc.\\1600 Amphitheatre Parkway\\Mountain View, CA 94043}
% The date on the letter, default is the date of compilation
\letterdate{\today}
% The title of the letter
% \lettertitle{Job Application}
% How the letter is opened
\letteropening{}
% How the letter is closed
\letterclosing{}
% Any enclosures with the letter
% \letterenclosure[Attached]{Curriculum Vitae}


%-------------------------------------------------------------------------------
\begin{document}

% Print the header with above personal informations
% Give optional argument to change alignment(C: center, L: left, R: right)
% \makecvheader[R]

% Print the footer with 3 arguments(<left>, <center>, <right>)
% Leave any of these blank if they are not needed
% \makecvfooter
% {\today}
% {Claud D. Park~~~·~~~Cover Letter}
% {}

% Print the title with above letter informations
% \makelettertitle

%-------------------------------------------------------------------------------
%	LETTER CONTENT
%-------------------------------------------------------------------------------
\begin{cvletter}

    \section{Experience}

    \subsubsection*{Describe a skill or knowledge you acquired recently that has been impactful for you. Why did you make this investment? What has the outcome been?}

    My most recent project I worked on was to create an active object in the Quantum Leaps QP Framework to act as a sort of simulation, where the goal was simply to receive calls, data and callbacks from other active objects and log or perform callbacks as appropriate. This was my first experience with Quantum Leaps software or frameworks, or any kind of big embedded framework. I was able to see how other services were implemented, giving me an idea of what a big embedded project would look like. In the process of learning how to use the framework and how it works I was also able to gain a deal of insight into the implementation of event-driven state machines in a real-time embedded framework.

    \subsubsection*{What new skill would you like to learn? Why do you think this is important or timely or interesting? Why do you think you will be good at it?}

    During my time at a previous job, I was lightly exposed to the world of time sensitive networking and the Precision Time Protocol (PTP) that another team was working on. I found the time sensitivity of it quite interesting and would have loved to dive into it. I think the time sensitivity of such systems is quite critical, and the experience gained from it would translate very well into other embedded domains, where performance and real time functionality can be crucial. I believe that kind of precision work could suit my detail-oriented approach well.

    \subsubsection*{Describe your experience as a developer on embedded Linux.}

    I have worked on a TCP/IP-based protocol on embedded Linux, which involved creating bash scripts that used some basic commands such as \emph{ifconfig}, \emph{ip link}, etc. to set up and manage the vlan that the protocol were to utilize. I also had some experience using sockets and file descriptors.

    \subsubsection*{Describe your experience with Linux kernel development and debugging.}

    I have not yet had the privilege to work with kernel development or debugging.

    \subsubsection*{Describe your experience of low-level boot processes and BIOS / firmware.}

    I have had some experience working with the boot process for an ESP32 device, which involved working with partitions and loading a custom bootloader.
    I have no developmental experience with BIOS, and am only familiar with configuring it and UEFI for the purposes of flashing Linux distributions.

    \subsubsection*{How do you address software performance in your coding practice?}

    Software performance is always a consideration for me when coding, and influences a lot of decisions concerning the use of data structures, algorithms, and, in some cases more importantly, third-party libraries; however, for most of my experience I could get away with "good enough" and trying to squeeze out more performance would not have been worth the time consumption.

    \subsubsection*{How do you prefer to drive documentation for your products?}

    I believe that good coding practices carry a lot of weight when it comes to developer documentation. If the code is not readable, there is very little that external documentation can do to alleviate that. Good code comes first, then documentation should exist to supplement it, cover gaps and shortfalls, and act as a guiding tool: where to look for the implementation of specific functionality and how to find related libraries and implementations.

    \subsubsection*{How do you think about and ensure quality in your software products?}

    In the development process, code that could start off as a rough draft of sorts, the bare minimum to achieve desired functionality, is meant to be revisited with one or more different mindsets. The code has to be looked over with possible bugs, shortfalls, and edge cases in mind. It also has to be looked over with a pruning mindset to identify what parts of the code actually belongs somewhere else, or what parts shouldn't even exist. It should also be looked over with the user in mind, ensuring ease of use and reducing surface area.

    \subsubsection*{Describe a case where it was very difficult to test code you were writing, but you found a reliable way to do it.}

    When writing code for an ESP32 microcontroller to flash a new bootloader from a separate STM32 microcontroller, I did not have access to the STM32 to be able to test this. The most reliable way to test this was to simulate sending packets via a Python application on a host PC.

    \subsubsection*{Describe your C/C++ software development experience to date. How would you rate your competency with C/C++?}

    I have three years working experience with C/C++, involving working on a TCP/IP-based protocol service on embedded Linux, as well as creating programs for ESP32. I have had ample experience working with C++ template metaprogramming, as well as embedded C++.

    \subsubsection*{Describe your experience in Golang and Python.}

    I have three years working experience with Python, which includes using Python to create testing scripts with Pytest, and many other applications to have had ample experience with Pythonic code.

    \subsubsection*{Which Linux middleware/user space stacks are you the most familiar with? For example gstreamer, libinput, audio subsystems etc}

    As mentioned before, I am only slightly familiar with \emph{ifconfig} and the \emph{ip} commands.

    \subsubsection*{What interesting syscalls are used by the “uname” binary? How did you find out?}

    Since \emph{uname} is part of the coreutils library, a quick look at the source code shows that it uses \emph{sysinfo} and \emph{sysctl} to get the processor architecture and hardware platform. It also uses the \emph{uname} syscall to retrieve kernel information.

    \subsubsection*{Have you created .deb or .rpm packages? Please describe your experience with Linux packaging.}

    I have not yet had the privilege of creating a package yet. I have only grazed the surface of using \emph{dpkg} to install an existing package.

    \subsubsection*{What kinds of software projects have you worked on before? Which operating systems, development environments, languages, databases?}

    I have worked with Python and C++ on embedded Linux running Ubuntu, and ESP32 microcontrollers. I have used WSL2 running Ubuntu for the majority of my working experience. I toyed around with using various Linux distributions as dev environments during my time at university, and have used both Windows and MacOS for development.


    \section{Education}

    \subsubsection*{How did you rank in your final year of high school in mathematics? Were you a top student? On what basis would you say that?}

    I was ranked \#1 for Mathematics in my high school class, achieving the highest score among my peers with a 94\% for my matric year.

    \subsubsection*{How did you rank in your final year of high school, in your home language? Were you a top student? On what basis would you say that?}

    I was in the top 10 for Afrikaans in my high school class, out of roughly 120 students.

    \subsubsection*{Please state your high school graduation results or university entrance results, and explain the grading system used. For example, in the US, you might give your SAT or ACT scores. In Germany, you might give your scores out of a grading system of 1-5, with 1 being the best.}

    In a percentage-based grading system, with 80\% or higher being considered achievement level 7, or an A, I achieved the following marks:

    \begin{itemize}
       \item Afrikaans Home Language: 83\%
       \item English Secondary Language: 83\%
       \item Mathematics: 94\%
       \item Information Technology: 87\%
       \item Physical Sciences: 84\%
       \item Life Sciences: 84\%
    \end{itemize}


    \subsubsection*{Can you make a case that you are in the top 5\% in your academic year, or top 1\%, or even higher? If so please outline that case. Make reference where possible to standardised testing results at regional or national level, or university entrance results. Please explain any specific grading system used.}

    I was in the top 5\% for my graduating class with an average of 85.8\%, with 6 distinctions (subjects above 80\%).

    \subsubsection*{What sort of high school student were you? Outside of class, what were your interests and hobbies? What would your high school peers remember you for?}

    I was a quiet kid at school, who mostly enjoyed spending his free time on his family's computer, playing games and messing around with things.

    \subsubsection*{Which university and degree did you choose? What other universities did you consider, and why did you select that one?}

    I chose to study B.Eng. Computer and Electronic Engineering at the North-West University Potchefstroom Campus. I chose this university because they were ECSA (Engineering Counsil of South Africa) accredited and my degree would have international recognition.

    \subsubsection*{Overall, what was your degree result and how did that reflect on your ability? Please help us understand the grading system for your results.}

    I formally satisfied all the requirements for my degree, achieving multiple distinctions for various modules, such as 78\% for \emph{Computer Engineering I}, 75\% for \emph{Electronics I}, 96\% for \emph{Programming for Engineers I (C++)}, 78\% for \emph{Object-Oriented Software Development}, 78\% for \emph{Introductory Algebra and Analysis I}.

    I believe my results reflect my ability to deliver statisfactory work under immense pressure as well as do exceedingly well when given the time and accomodation.

    \subsubsection*{During all of your education years, from high school to university, can you describe any achievements that were truly exceptional?}

    In my final year of high school, I graduated in the top 5\% of my class. During my first year at university I was eligible to join the Golden Key International Honours Society for being in the top 15\% of my class.

    \subsubsection*{What leadership roles did you take on during your education? Did you conceive of, and drive to completion, any initiatives outside of your required classwork?}

    I did not take on any leadership roles and mainly focused my time and attention on academics and furthering my knowledge.


    \section{Career development}


    \subsubsection*{How would you describe your experience as a professional software engineer?}

    I have had the opportunity to work in a small team to deliver critical services to our clients. I have also had the opportunity to work as a mostly solo developer working on ESP32 functionalities for an established international company.


    \subsubsection*{What are your strengths as a software engineer?}

    I have a passion for continual learning and problem solving and have a demonstrated ability to swiftly adapt to evolving project requirements. I am detail-oriented and keen to dive into the minutiae of interesting topics. I take pride in my work and always strive to deliver excellent results.

    \subsubsection*{What is your proudest success as an engineer?}

    My proudest success to date was being promoted to team lead of my small team after only one year at the company.

    \subsubsection*{What would you like to achieve in your career and skills development?}

    I want to learn more about various interesting fields. I also want to find my niche, a topic where I can become an expert in, and use the broad knowledge that I have gained to excel at it.

\end{cvletter}

\end{document}
